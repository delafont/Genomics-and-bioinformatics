
\documentclass[a4paper,11pt]{article}

\usepackage[latin1]{inputenc}
\usepackage[T1]{fontenc}
%\usepackage[francais]{babel}
\usepackage{bbm} %math chars
\usepackage{amsmath}
\usepackage{indentfirst}
\usepackage{fullpage} %minimizes the default margins
\usepackage{url}
\usepackage{graphicx}
\usepackage[center,footnotesize]{caption} %options des legendes des graphes
\usepackage[section]{placeins} %place les figures d'une section avant le debut de la suivante
\usepackage{subfig} %a) b) c)

\title{Series 1}
\date{September 20, 2011}
\author{Genomics and bioinformatics - Week 1}


\begin{document}
\maketitle


\section{UCSC genome browser}
Go to the address \url{http://genome.ucsc.edu/}. Click on ``Genome browser'', select a species 
and visualize its genomic content; zoom in and out.

\section{Exercise}
To do this exercise, install R and Python on your computer (see below).

Use as much of the documentation given below as needed to obtain the result we want here, using successively R and Python.

Testing file: download genes\_expression\_100.txt at \url{testing file}. It is an RNA-seq output file in .txt tab-delimited format - 100 lines of the form\\
(Gene name <tab> Expression in condition 1 <tab> Expression in condition 2).

Using Python, then R, do: 
\begin{enumerate}
\item read the file and extract gene names and associated numbers;
\item compute the ratios between the two numerical columns; take $log_{2}$ of the result;
\item compute the geometric means between the two numerical columns; take $log_{10}$ of the result;
\item write these in a new text file: (GeneName <tab> $log_2$(ratio) <tab> $log_{10}$(mean));
\item plot ratios vs means (use matplotlib for Python).
\end{enumerate}

\section{R}
\subsection{Installation} 
Download R 2.13 at \url{http://stat.ethz.ch/CRAN/}

\subsubsection{Windows}
For both R and RStudio: during the installation, choose where you want to install your folder; then you can run it clicking on the .exe file located in the bin/ sub-folder. 

\subsection{Tutorial}
R tutorial: \url{http://cran.r-project.org/doc/manuals/R-intro.pdf}
    %Example tutorial:
    %R is for statistics
    %R is most useful when you already have data
    %- open a data file: read.table('data.txt') > data.frame
    %- extract columns, lines: data[,1], data[1,], data[1]
    %- class()
    %- create vectors: c()
    %- use some functions such as mean, var, median
    %- graphics: histograms, boxplots, pairs(), plot()
    %- write to a file: write.table(data, 'output.txt')
    %- do the exercise 
    
If you need help:
\begin{enumerate}
\item Read the tutorial
\item Use the ? or help() R commands
\item Use Google
\item Ask us.
\end{enumerate}

\subsection{Reference documentation}
\url{http://cran.r-project.org/doc/manuals/refman.pdf}


\section{Python}

\subsection{Installing}
Download the Enthought Python distribution at \url{http://www.enthought.com/products/edudownload.php}

Open a console* and type ``ipython''
(*on Windows, open the ``Start'' menu and type ``cmd'' in the search field).

\subsection{Tutorial}
Python tutorial: \url{http://docs.python.org/tutorial/} : read 3.1 to 4.6, 5.5 and 7.2.

For Numpy: \url{http://www.scipy.org/Tentative_NumPy_Tutorial}
    %Example tutorial:
    %- use it as a calculator > int, float
    %- print "Hello" > strings
    %- create a script.py file with several commands and run it : execfile('script.py')
    %- create lists, dictionaries, use range()
    %- create a for loop
    %- insert if statements
    %- play with loops, lists and dicts
    %- open a file: f = open('data.txt'), f.close()
    %- iterate over the lines and print them
    %- extract elements from a line: split(); \n and \t tags
    %- write to a new file: g.write()
    %- do the exercise.
    %Some more tutorials: http://wiki.python.org/moin/BeginnersGuide/Programmers 

\subsection{Reference documentation}
\url{http://docs.python.org/library/index.html}
\end{document}