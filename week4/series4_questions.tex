\documentclass[a4paper,11pt]{article}
\usepackage[latin1]{inputenc}
\usepackage[T1]{fontenc}
\usepackage{bbm}
\usepackage{amsmath}
\usepackage{indentfirst}
\usepackage{fullpage}
\usepackage{url}
\usepackage{geometry}
\geometry{verbose,tmargin=3cm,bmargin=2cm,lmargin=2cm,rmargin=2cm}
\usepackage{graphicx}
\usepackage{epstopdf}
\usepackage[center,footnotesize]{caption}
\usepackage[section]{placeins}
\usepackage{subfig}
\DeclareRobustCommand{\greektext}{%
  \fontencoding{LGR}\selectfont\def\encodingdefault{LGR}}
\DeclareRobustCommand{\textgreek}[1]{\leavevmode{\greektext #1}}
\DeclareFontEncoding{LGR}{}{}
\DeclareTextSymbol{\~}{LGR}{126}
\title{Series 4}
\date{}
\author{Genomics and bioinformatics - Week 4 - October 9, 2011}
\begin{document}
\maketitle


\section{Sequence alignment: Needlman-Wunsch}
In this exercise you will manually perform a global alignment of two sequences using the
Needleman-Wunsch algorithm and based on the following scoring scheme: for $X,Y$ in $\{A,T,G,C,-\}$, 
$$
M(X,Y) = \left\{ 
\begin{array}{l}
	+2 \quad\text{if}\quad X = Y \\
	-1 \quad\text{if}\quad X \neq Y \\
	-2 \quad\text{if}\quad X = - \quad\text{or}\quad Y = -
\end{array} \right.
$$
Sequence 1: \texttt{GAATTCAG}\\
Sequence 2: \texttt{GGATCG}.
\vspace{0.5cm}
\begin{center}
\includegraphics[width=0.8\textwidth]{matrix.png}
\end{center}
\vspace{0.5cm}

The best alignment is: ........\\

\newpage 

\section{Pair Hidden Markov Model}

In this exercise, we will construct a pair Hidden Markov Model for
the same sequences as in the first exercise and align them using the
path with maximum probability. The maximum probability path and the 
corresponding alignment are calculated by an algorithm called the Viterbi Algorithm. 
You will see through the exercise that the Viterbi algorithm is actually equivalent
to the Needleman-Wunsch algorithm.


\begin{figure}[h]
\begin{center}
\includegraphics[width=0.5\textwidth]{HMM.jpg}
\caption{Pair Hidden Markov Model}
\label{fig:HMM}
\end{center}
\end{figure}


The Pair HMM consists of the following parameters (see fig.~\ref{fig:HMM}):

\vspace{0.5cm}

Three states:

State M matches one letter from each sequence.

State I (Insertion) inserts a gap to the second sequence.

State D (Deletion) inserts a gap to the first sequence.

\vspace{0.5cm}

Emission probabilities:

$p(x,y)$ = probability of emitting a pair of characters {[}x,y{]}.

$q(x)$ = probability of emitting a pair of characters {[}x,-{]} (insertion).

$q(y)$ = probability of emitting a pair of characters {[}-,y{]} (deletion).

\vspace{0.5cm}

Transition probabilities:

$\delta$ = probability of opening a gap

$\varepsilon$ = probability of extending a gap

\vspace{0.5cm}

To make correspondence to the Needleman-Wunsch algorithm with the scores given in Exercise 1,


$$S(x,y)=log_2\frac{p(x,y)}{p(x)\; p(y)} \quad , \quad d=-log(\delta) \quad ,$$


where

S(x,y) = 1 if $x=y$, 

S(x,y) = -1 if $x\neq y$,

d = e = -2 for gap penalty.

\newpage
The Viterbi algorithm goes through the three steps:\\


Step 1: Initialization:\\

$V_M(0,0):=0$

$V_D(0,0):=-\infty$

$V_I(0,0):=-\infty$

$V_*(-1,j)=V_*(i,-1):=-\infty \;$ ($_*$ accounts for either M, D or I)
\vspace{0.5cm}

Step 2: Recursion:
\begin{eqnarray}
&&
V_M(i,j) =S(x_{i},y_{j})+max 
	\left\{ \begin{array}{l}
	 V_M(i-1,j-1) \\
     V_D(i-1,j-1) \\
     V_I(i-1,j-1)
    \end{array} \right.\nonumber\\
&&
V_D(i,j) =max \left\{ 
    \begin{array}{ll}
     V_M(i-1,j)-d \\
     V_D(i-1,j)-e 
    \end{array} \right.\nonumber\\
&&
V_I(i,j) =max \left\{ 
    \begin{array}{ll}
     V_M(i,j-1)-d\\
     V_I(i,j-1)-e
    \end{array} \right.\nonumber
\end{eqnarray}


Step 3: Termination:

$$V_E=max(V_M(n,m),V_D(n,m),V_I(n,m))$$


\begin{enumerate}
\item Deduce the emission probability matrix and the transition probabilities of the HMM.
\item Use the algorithm as shown above to generate the three matrices for Match(M), Deletion(D) and Insertion(I).
\item Deduce the alignment based on the three matrices.
\end{enumerate}

\newpage

Matrix $V_M$:
\begin{center}
\includegraphics[width=0.8\textwidth]{matrix.png}
\end{center}
\vspace{1.5cm}

Matrix $V_D$:
\begin{center}
\includegraphics[width=0.8\textwidth]{matrix.png}
\end{center}
\vspace{1.5cm}

\newpage 

Matrix $V_I$: 
\begin{center}
\includegraphics[width=0.8\textwidth]{matrix.png}
\end{center}
\vspace{0.5cm}

Backtracking matrix:
\begin{center}
\includegraphics[width=0.8\textwidth]{matrix.png}
\end{center}
\vspace{0.5cm}

The possible alignments are: ........................


\end{document}
