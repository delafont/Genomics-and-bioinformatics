\documentclass[a4paper,11pt]{article}
\usepackage[latin1]{inputenc}
\usepackage[T1]{fontenc}
\usepackage{bbm}
\usepackage{amsmath}
\usepackage{indentfirst}
\usepackage{fullpage}
\usepackage{url}
\usepackage{graphicx}
\usepackage[center,footnotesize]{caption}
\usepackage[section]{placeins}
\usepackage{subfig}

\title{Series 4}
\date{October 11, 2011}
\author{Genomics and bioinformatics - Week 4}

\begin{document}
\maketitle

\section{Sequence alignment}
The Needlman-Wunsch algorithm uses a method called ``dynamic programming''. This is a very general programming technique. It involves three main steps,
\begin{enumerate}
\item Initialization
\item Scoring (matrix fill)
\item Alignment (backtracking)
\end{enumerate}

In the first exercise of this session you will manually perform a global alignment of two sequences based on the following scoring scheme,

\emph{Match:}  \texttt{+1}, \emph{Mismatch:} \texttt{-1}, \emph{Gap:} \texttt{-2}\\

Sequence 1: \texttt{GAATTCAGA}

Sequence 2: \texttt{GGATCGA}.

\begin{center}
\includegraphics[width=0.8\textwidth]{matrix.png}
\end{center}

Solution:

\end{document}
